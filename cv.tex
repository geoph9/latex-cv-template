% !TEX root = Simple-CV.tex
%-------------------------------------------------------------------------------------------------------
% Packages
%-------------------------------------------------------------------------------------------------------
\usepackage[latin1]{inputenc}
\usepackage[T1]{fontenc}
\usepackage[english]{babel}
\usepackage{fontawesome}
\usepackage{datetime}
\usepackage[usenames,dvipsnames]{xcolor}
\usepackage[colorlinks=true, urlcolor=ColorTwo]{hyperref}
\usepackage{tikz}
\usepackage{hyperref}
\usepackage{setspace}
\usepackage{graphicx}
\usepackage{enumitem}
\usepackage{sectsty}
\usepackage{multicol}
\usepackage{adjustbox}


%----------------------------------------------------------------------------------------
%	FONTS
%----------------------------------------------------------------------------------------

\usepackage{fontspec} % Required for specifying custom fonts in XeLaTeX

\setmainfont[Color=primary, Path = fonts/lato/,BoldItalicFont=Lato-RegIta,BoldFont=Lato-Reg,ItalicFont=Lato-LigIta]{Lato-Lig} % The primary font for content text; defines bold, italic and bold-italic as well

\setsansfont[Scale=MatchLowercase,Mapping=tex-text, Path = fonts/raleway/]{Raleway-ExtraLight} % The font used where \sfffamily is called


%----------------------------------------------------------------------------------------
%	MAIN HEADING COMMAND
%----------------------------------------------------------------------------------------

\definecolor{headings}{HTML}{6A6A6A} % The color of the large sections
\colorlet{shadecolor}{gray!40}

\newcommand{\namesection}[3]{ % Defines the command for the main heading
    \centering{ % Center the name
        \fontsize{18pt}{40pt} % Font size
        \fontspec[Path = fonts/lato/]{Lato-Hai}\selectfont #1 % First name font
        \fontspec[Path = fonts/lato/]{Lato-Lig}\selectfont #2 % Last name font
    } \\[1pt] % Whitespace between the name and contact information
    \centering{ % Center the contact information
        \color{headings} % Use the headings color
        \fontspec[Path = fonts/raleway/]{Raleway-Medium}\fontsize{11pt}{14pt}\selectfont #3\\
        \makebox[280pt]{\color{shadecolor}\rule{350pt}{0.8pt}} % Horizontal rule
    } % Contact information font
    % \noindent\makebox[\linewidth]{\color{shadecolor}\rule{\paperwidth}{0.8pt}} % Horizontal rule
    \vspace{-8pt} % Reduce whitespace after the rule slightly
}


%-------------------------------------------------------------------------------------------------------
% Layout
%-------------------------------------------------------------------------------------------------------
\pagenumbering{gobble}
\renewcommand{\baselinestretch}{1.45}
\setlength{\parindent}{0pt}

%
% Color theme
%
\definecolor{ColorOne}{RGB}{0,110,140} 	% Blue
\definecolor{ColorTwo}{RGB}{120,0,120} 	% Mauve
%\definecolor{ColorTwo}{RGB}{140,100,0} 	% Gold

\sectionfont{\color{ColorOne}} 
\subsectionfont{\color{ColorOne}} 

% \newcommand{\MyDummySection}{\color{ColorOne}}
\newcommand*{\MyDummySection}[1]{{\fontsize{14pt}{1em}\color{ColorOne}{\textbf{#1}}}}

%
% Vertical line
%
\newcommand{\MyVerticalRule}{%
	\textcolor{ColorOne}{\rule{0.4pt}{0.885\textheight}}
}
%
% Horizonal line
%
\newcommand{\MyHorizontalRule}{%
% 	\textcolor{ColorOne}{\rule{1pt}{0.85\textheight}}
    \vspace{-15pt}
	\textcolor{ColorOne}{\rule{515pt}{0.4pt}} % Horizontal rule
}

%
% Update
%
\newcommand{\LastUpdate}{%
\vfill
\centering \small
\textcolor{ColorOne}{Last updated: \monthname,~\the\year.}
}

%
% Skip
%
\newcommand{\MySkip}{
\vskip12pt
}

\newcommand{\MyBullet}{
\textcolor{ColorOne}{$\circ$}
}

%
% Format hyperrefs
%
\newcommand{\myhref}[2]{%
\href{#1}{\textcolor{ColorTwo}{#2}}
}
%
% Format skill bullets
%
\newcommand{\SkillBull}[1]{%
\textcolor{ColorTwo}{#1}
}



% Taken from 2nd resume template
\newcommand*{\subsectionstyle}[1]{{\fontsize{12pt}{1em}\bodyfont\scshape\textcolor{text}{#1}}}
\newcommand{\cvsubsection}[1]{%
  \vspace{\acvSectionContentTopSkip}
  \vspace{-3mm}
  \subsectionstyle{#1}
  \phantomsection
}

\newcommand{\resumeItem}[1]{
  \item\small{
    {#1 \vspace{-2pt}}
  }
}


\newcommand{\resumeSubheading}[4]{
  \vspace{-2pt}\item
    \begin{tabular*}{0.97\textwidth}[t]{l@{\extracolsep{\fill}}r}
      \textbf{#1} & #2 \\
      \textit{#3} & \textit{\small #4} \\
    \end{tabular*}\vspace{-7pt}
}


\newcommand{\resumeSkillheading}[2]{
  \vspace{-2pt}\item
    \begin{minipage}[c]{0.3\linewidth}
        \begin{tabular}{l}
          \textbf{#1} \\
        \end{tabular}
    \end{minipage}
        % \vfill\null \columnbreak  % Break column for new section
    \begin{minipage}[c]{0.7\linewidth}
        \begin{tabular}{l}
          #2 \\
        \end{tabular}
    \end{minipage}
        \newline
    % \end{multicols}
    \vspace{-45pt}
}


\newcommand{\resumeSubSubheading}[2]{
    \vspace{-2pt}\item
    \begin{tabular*}{0.97\textwidth}{l@{\extracolsep{\fill}}r}
      \textit{\small#1} & \textit{\small #2} \\
    \end{tabular*}\vspace{-7pt}
}


\newcommand{\resumeEducationHeading}[6]{
  \vspace{-10pt}
    \begin{tabular*}{0.97\textwidth}[t]{l@{\extracolsep{\fill}}r}
      \textbf{#1} & #2 \\
      \textit{\small#3} & \textit{\small #4} \\
    \end{tabular*}\vspace{-5pt}
}


\newcommand{\resumeProjectHeading}[2]{
    \vspace{-2pt}\item
    \begin{tabular*}{0.97\textwidth}{l@{\extracolsep{\fill}}r}
      \small#1 & #2 \\
    \end{tabular*}\vspace{-7pt}
}


\newcommand{\resumeOrganizationHeading}[4]{
  \vspace{-2pt}\item
    \begin{tabular*}{0.97\textwidth}[t]{l@{\extracolsep{\fill}}r}
      \textbf{#1} & \textit{\small #2} \\
      \textit{\small#3}
    \end{tabular*}\vspace{-7pt}
}

\newcommand{\resumeSubItem}[1]{\resumeItem{#1}\vspace{-4pt}}

\renewcommand\labelitemii{$\vcenter{\hbox{\tiny$\bullet$}}$}

\newcommand{\resumeItemListStart}{\begin{itemize}}
\newcommand{\resumeItemListEnd}{\end{itemize}\vspace{-5pt}}